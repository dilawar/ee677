\section{VLSI Design Automation}
\label{sec:vlsi_design_automation}

  These problems can be classified into following subsections. We also hint at
  possible approach to their solutions.

\subsection{Circuit partitioning}

  Given a circuit, we model it as a graph. Each node represents a circuit
  element and edge represent the interconnects. For example consider the
  following figure.

\begin{figure}[h]
\centering
\subfloat[A circuit]{
  \begin{tikzpicture}[circuit logic US, scale=0.8]
    \node (i0) at (0,0) {}; 
    \node (i1) [below = of i0]{};
    \node (i2) [below = of i1]{};
    \node [nand gate, right=of i0] (n1) {5};
    \node [xor gate, right=3cm of i1] (x1) {4};
    \node [nand gate, right=of i2] (n2) {6};
    \node [and gate, right=of x1] (a1) {2};
    \node [not gate, right=of a1] (in1) {1};
    \node [not gate, rotate=90, below left=of a1] (in2) {3};

    %\draw (i0.east) -- ++(right:3mm) |- (n1.input 1);
    \draw (n1.input 1) -- ++(left:10mm);
    \draw (i1.east) -- ++(right:5mm) |- (n1.input 2);
    \draw (i1.east) -- ++(right:5mm) |- (n2.input 1);
    %\draw (i2.east) -- ++(right:3mm) |- (n2.input 2);
    \draw (n2.input 2) -- ++(left:10mm);

    \draw (n1.output) -- ++(right:5mm) |- (x1.input 1);
    \draw (n2.output) -- ++(right:5mm) |- (x1.input 2);
    \draw (x1.output) -- ++(right:3mm) |- (a1.input 1);
    \draw (in2.output) -- ++(right:3mm) |- (a1.input 2);

    \draw (a1.output) -- ++(right:3mm) |- (in1.input);
    \draw (in1.output) -- ++(right:3mm);
    \draw (in2.input) -- ++(down:3mm);
  \end{tikzpicture}
 }
 \subfloat[Modeled as graph] {
  \begin{tikzpicture}
    [every node/.style={draw, circle}]
    \node (n5) {5};
    \node (n6) [below =of n5] {6};
    \node (n4) [right =of n5, yshift=-7mm] {4};
    \node (n2) [right =of n4] {2};
    \node (n3) [below =of n2, yshift=4mm] {3};
    \node (n1) [right =of n2] {1};

    \draw[-] (n5) -- (n6);
    \draw (n5) -- (n4);
    \draw (n4) -- (n6);
    \draw (n4) -- (n2);
    \draw (n2) -- (n3);
    \draw (n2) -- (n1);

  \end{tikzpicture}
  }
\end{figure}

  Now one can state a generic partitioning problem.

  \begin{problem}[40]

    Given a graph $G(V,E)$ where each vertex $v \in V$ has a size $s(v)$, and
    each edge $e \in E$ has a weight $w(e)$, the problem is to divide the set
    $V$ into $k$ subsets $V_1,\ldots,V_k$ such that an objective function is
    optimized, subject to certain constraints.

  \end{problem}

  One can formulate various constraints. Better known among them the listed
  below.

  \begin{description}
  \item [Bounded size] Assume that a single chip does not allow more than 6
  gates on average and your design has 100 of gates. Thus, you need to partition
  your circuit into at least 17 subsets, each of them is not having more than 6
  gates. Wouldn't be great if each partition have equal number of gates?

  \item[Minimize external wiring] Another situation is that we have many
  packages but there is a need to connect these packages through external wires.
  It is hightly desirable to minimize the lenght of external wiring (why?). 

  \end{description}

  One of the most popular algorithm to partitioning problem is due to
  \textbf{Keninghan and Lin} which finds a partition and improves it further.
  A heuristic is due known as \textbf{Fiduccia Mattheyses}. Some of the most
  powerful algorithm are based on a technique known as \textbf{simulated
  annealing}.
  
\subsection{Placement and routing problem}

  TODO : Give a small description and points to well known algorithms.

  \begin{description}
  \item [Grid routing]
  \item [Channel routing]
  \item [Layout generation]
  \item [Floorplanning]
  \end{description}

\subsection{Verification, validation and testing}

Following is from course notes provided by Prof. Madhav P. Desai, Verification
of VLSI circuits, 2010.

\begin{quotation}
"
Testing, verification and validation are critical processes in ensuring that the
circuit that is being manufactured meets the requirements that motivated its
design.

\begin{itemize}

\item By \textbf{testing}, we normally refer to the process used to determine
that the manufactured circuit is functional. Every manufactured circuit needs to
be tested, and the result of the test should, with a high degree of confidence,
determine that the circuit is functional.

\item By \textbf{verification}, we normally refer to the process used to confirm
that refinements in the design process are consistent with the circuit
specifica- tion. For example, when a logic circuit is implemented using
transistors, we need to verify that the transistor network is equivalent to the
logic circuit which it is supposed to implement. This is done at each refinement
step in the design process.

\item By \textbf{validation}, we normally refer to the process used to confirm
that a functional manufactured circuit will fulfill the requirements to which it
was designed. This usually involves the construction of a prototype and a test
setup which mimics reality to the maximum extent possible.  

\end{itemize}

Typically, the effort needed to implement these processes in the design of
complex circuits is substantial; a major fraction of the total product
development effort is contributed by the test, verification and validation
processes.  
"
\end{quotation}

Following are the major problems in testing.

\begin{itemize}

\item Manufacturing defects and fault models.  
\item The formulation of the test problem.  
\item The use of fault simulation in identifying fault coverage.  
\item Test pattern generation for combinational gate-level circuits using the
stuck-at fault model.  
\item Testing of sequential circuits.  
\item The use of scan techniques to test sequential circuits.  
\item Built-in self-test (BIST).  
\item Error correction and schemes for constructing fault-tolerant circuits. 
\end{itemize}

Verification is often done using simulation but this is not enough these days.
Use of formal verification is rising.
\begin{description}

\item[Formal Verification of combinational circuits] Given
a circtuit, prove some properties holds on it. To do it, we may use Binary decision diagrams
or Satisfiability solvers. 

\item [Formal verification of sequential circuits]  Given a state machine, prove
that some properties holds at a given time. Prove that given two designs or two
state machines are equivalent. State exploration techniques, model
checking techniques

\end{description}

\paragraph{Fault diagnosis}

  Refer to lecture notes. To be updated.

