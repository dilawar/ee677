\section{Boolean Logic}

\begin{quotation}
 \begin{flushright}
 'Contrariwise', continued Tweedledee, 'if it is so, 
 it might be; and if it were 
 so, it would be;  but as it isn't, it ain't. That's logic. \\
 -- \textbf{Lewis Carroll}, Through the Looking Glass (1871) 
 \end{flushright}
\end{quotation}
  
  \paragraph{A note from history} Modern Boolean expressions are written
  differently from what were originally devised by George Boole.  He thought
  that 0 and 1 could be used to construct a   calculus of logical reasoning; in
  his \textit{An Investigation of Laws of Thoughts} where he wrote,
  \begin{quote} Let us conceive, then, of an Algebra in which the symbols x, y,
    z, \&c admit indifferently of the values of 0 and 1, and of these values
    alone.  \end{quote}.  
  It is surprising that he never dealt with the "logical or" operation $+$; and
  he restricted himself to ordinary arithmetic on 0 and 1. In his system, $1+1$
  was 2 and not 0. He took pain to deal with the case when both of terms in
  $x+y$ were 1; he wrote $x+(1-x)y$ to ensure that result of this disjunction
  would never be equal to 2. He also called his $+$ operator both "or" and "and"
  in his paper. These days this practice will be strange if not unacceptable.
  But when we realize that his usage was in fact normal English, it looks
  reasonable. As Knuth explains in [The art of computer programming, 4(A)], we
  say, for example, that "boys \textbf{and} girls are children," but "children
  are boys \textbf{or} girls".
  
 \subsection{Boolean Satisfiability problem}

  A Boolean \footnote{In old times, Boolian} function is said to be
  \textbf{satisfiable} if it is not identically zero - i.e. it has at least one
  implicant. To find an \emph{efficient} way to decide whether a given Boolean
  function is satisfiable or not is the most famous problem in computer science.
  To be precise, we ask : \emph{Given a Boolean formula of length $N$ and test
  it for satisfiability, always giving the correct answer after performing at
  most $N^{O(1)}$ steps?} [Knuth, TAOP 4(A)].

  Granted that testing for satifiability in general is tough. In fact, it is
  difficult even in simplified form; a conjuctive normal form that has only
  three \textit{literals} in each clause : 
  \begin{equation}
  \label{eq:3sat}
  f(x_1,\ldots,x_n) = (t_1+u_1+v_1).(t_2+u_2+v_2)\ldots(t_m+u_m+v_m)
  \end{equation}

  Here each $t_j, u_j$, and $v_j$ is $x_k$ or $x'_k$ for some $k$. This form is
  called \textbf{3CNF} and problem of deciding satisfiability of these formulas
  is called \textbf{3SAT}. We'll restrict our short attention span to 3SAT
  problems for any general SAT problem can be reduced to a 3SAT problem.

  \paragraph{Example}

  The shortest unsatisfiable Boolean formula in 3CNF known to humanity is
  $(x+x+x).(x'+x'+x')$. This is lame for it is of no challenge to the intellect of
  an initiated. 

  \begin{remark}

  Can you think of a program which generates Boolean functions in 3CNF forms
  which are not satisfiable? 
  
  \end{remark}

  When we assume that three literals in all clauses in equation \ref{eq:3sat} are
  differnt then we run into troubled water. Each clause will rule out exactly
  1/8 possibilities (why?). Thus any such formula with 7 clause is automatically
  satifiabile!
  
  The shortest interesting formula in 3CNF has 8 clause and is due to R. L.
  Rivest which is following unless I made some mistake in copying it down.

  \begin{equation}
  \label{eq:shortest_3sat}
  \begin{gathered}
  (x_2+x_3+x'_4).(x_1+x_3+x_4).(x'_1+x_2+x_4).(x'_1+x'_2+x_3). \\
  (x'_2+x'_3+x_4) .(x'_1+x'_3+x'_4).(x_1+x'_2+x'_4).(x_1+x_2+x'_3)
  \end{gathered}
  \end{equation}

  \paragraph{Algorithm and data-structures}

  Boolean function will be given as expression-trees stored in \texttt{graphml}
  format.  Binary decision diagrams (BDD) can also be emplyoed in solving this
  problem \ref{sec:data_structure}. I am not aware of any algorithm which
  generates satisfiable 3CNF functions efficiently. It is very hard to give a
  list of function which are unsatisfiable. For example, look at this
  \href{http://cstheory.stackexchange.com/questions/8117/minimum-unsatisfiable-3-cnf-formulae}{discussion}.

  One of the most widely used algorithm to solve this problem is 
  Davis Putnam Logemann Loveland algorithm [DPLL].

    \begin{problem}[30] 
      Implement DPLL algorithm.
    \end{problem}

    \begin{remark}

    There are many SAT solvers. There is also an annual competetion where SAT
    solvers compete for efficiency. You can find many example problems. See
    \href{http://www.satcompetition.org/}{this webpage}.

    The standard format to these sat solver is DIMACS format. You can also use
    the same format.

    \end{remark}

    One might as well do the following in her spare time.

    \begin{problem}[20]
      Convert Boolean expression tree into DIMACS file.
    \end{problem}

  \begin{comment}
    Make sense of Tweedledee comment quoted at the begining of this section. 
    Lewis Carroll was a logician, so it is likely that this comment is logically
    consistent. 
    
    One solution is due to C. Sartena. Tweedledee is describing the implication
    $x \implies y$. What he actually wants to say is "Contrariwise, if y is so,
    x might be, and if x were so, y would be; but as y isn't; x ain't. Thats
    logic!".  
    
  \end{comment}


\subsection{Graph algorithms}
\label{sec:graph_algorithms}

In addition to the algorithms discussed in the classroom, following algorithms
can also be taken up for the project.

\paragraph{Graphs and matrix} 

Recall from network theory that a electrical network can be represented by a
graph. Graph was usually accompanied by its incidence matrix.

  \begin{problem}[15]
    Given a graph, compute its adjacency matrix, Laplacian matrix, and distance
    matrix. Also compute its eigenvalues.
  \end{problem}

\begin{remark}
  R. B. Bapat book \textbf{Graphs and Matrices} is a great introductory book.
  Students who are interested in graph theory may like to buy it for future
  reference.
\end{remark}

\paragraph{Graphs and circuits}

  Also recall that we constructed admittance matrix $A$ and then we solved
  $Ax=b$ to solve the circuit. 

  \begin{problem}[30]
     
     Given a ngspice netlist (R, V, I only) construct a graph which represents
     this circuit where each edge represents a device (R, V, I). Generate
     appropriate matrices to solve this circuit using modified nodal-analysis
     (MNA).

  \end{problem}

  Ngspice becomes very slow with nodes over 20-30 thousands. It was designed to
  use sparsed matrix techniques which does not scale over large circuits nicely.
  Another method $NAL-NBK$ is discussed in Prof. H. Narayana book \textbf{Submodular
  function and electrical network}.

  \begin{problem}[35] 
  
  Given a netlist (R,V,I), generate a network graph and solve it using $NAL-NBK$
  method. You can also explore the possibility of extending BITSIM with Prof.
  Patkar.  
  
  \end{problem}
    

\paragraph{Partitioning graphs} 

  Partitioning of graph is very important. Prof.  Patkar Ph.D. thesis was in
  the area of partitioning. You can request him to deliver an overview of
  partitioning techniques in a separate lecture.  Partitioning of graph can be
  done for various purposes. As far as computation is concerned, one major
  philosophy behind partitioning is \textbf{divide-and-conquer} such that each
  partitioned problem becomes 'independent' enough to be solved on different
  processor. For an example see chapter on 'multi-port decomposition" in Prof.
  Narayanan book.

  You are encouraged to partition a graph as you like (with some justification).
  Two of the following partition criteria, however, are too important to miss. 

  \begin{problem}[15]
  Given a graph, partition it into its strongly connected components,
  1-connected components, and 2-connected components.
  \end{problem}

  \begin{problem}[25]
  Given a simple connected graph, partition all vertices of it into smallest
  possible number of disjoint, independent sets. An independent set is a set
  of vertices such that no two vertices in the set are adjacent.
  \end{problem}

  \textbf{Some more problems are listed in section
  \ref{sec:vlsi_design_automation}}.

\paragraph{Matching TAs with courses}

  Here is one practical problem. Each year, EE department asks its Teaching
  Assistants to give three preferences of courses which they would like to serve
  as Teaching Assistant.

  \begin{itemize}
    \item There are $n$ TAs in the department and there are $m$ courses running.
    \item Each TA has a credit rating between 0 and 5.
    \item Each course has certain requirement of TA credits e.g. EE677 needs
    TAs worth 20 credits.
    \item Each TA belongs to any of the following 3 category : DD, MTECH, RS.
    \item Each category of TAs has its average rating.
  \end{itemize}

  Construct a bipartite graph with TAs on left and on the right hand side we
  have courses. Draw and edge between a TA and a course if and only if, he has
  given that course as his preference. Assign a weight on the edge according to
  the preference for example if 10407007 picks EE677 as his first choice, EE309
  as second and EE721 as his third choice then there are three edges going from
  10407007 to EE677, EE309, EE721 with edge weight-age of 10, 7, 3 (some other
  numbering is also possible).

\begin{problem}[45]
  Allot each TA a course such that sum of edges in matching is maximised. Make
  sure each course gets TA's such that distribution of credit set is 'fair'. For
  example if EE677 which requires TAs worth 20 credit can be alloted 4 TA's
  worth 5 credit each (highest possible) which is a bad distribution of 10 TA's
  each worth 2 credit, this is equally bad. We want as many good TA's as bad one
  in the course.

  Explore the possibility of adding another constraint : there is at least one
  TA from two different category?

  Assign credit randomly to TA's to test your algorithm. 

  What if a TA wants a course eagerly but instructor has given a negative
  feedback about him in the past? Or he does not want him at all? Find a
  strategy which causes least heart-burn.
  \end{problem}

\paragraph{Network flow}

  Network flow algorithms does not look very glamorous at first reading but they
  are one of the most effective tool in taming combinatorial problems. For
  instance, bipartite graph matching, integer programming, as well as linear
  programming problems can be solved using network flow methods.

  \begin{problem}[35]  
 
    Implement \textbf{push-relabel} method to compute max-flow in a network. Use
    priority queues for iteration over vertices.
  
  \end{problem}

